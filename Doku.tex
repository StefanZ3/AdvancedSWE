% !TEX encoding = UTF-8 Unicode

% Beispiel für ein LaTeX-Dokument im Format "seminarvorlage"
\documentclass[ngerman]{seminarvorlage}
% ngerman = Deutsch in neuer Rechtschreibung, alternativ english
\usepackage{babel} % automatische Sprachunterstützung

\usepackage[utf8]{inputenc} % Kodierung der Non-ASCII-Zeichen
\usepackage[T1]{fontenc} % Moderne Fonts, Trennung von Wörtern mit Umlauten
\usepackage{cleveref} % für bequeme Referenzen, siehe \cref unten

\usepackage{embrac}% upright brackets in emphasised text, () and [], empfohlen

%\usepackage{biblatex}% besser als bibtex, aber dann biber statt bibtex benutzen

\begin{document}

% Unbedingt angeben: Titel, Autoren
% Freiwillig: Adresse, E-Mail
\title{Warum keiner mehr agil arbeiten will}
\numberofauthors{2}
\author{
  \alignauthor Christoph Weber\\
    \email{i22037@hb.dhbw-stuttgart.de}
  \alignauthor Stefan Zimmerer\\
    \email{i22040@hb.dhbw-stuttgart.de}
}

\maketitle% Titelangaben produzieren, aber kein Inhaltsverzeichnis, \tableofcontents funktioniert nicht!
\newpage

\abstract{
}

\keywords{.}

% Section-Überschriften werden automatisch in GROSSBUCHSTABEN gesetzt
\section{Einleitung}

In einer sich rasant wandelnden Welt sind agile Methoden für Unternehmen unverzichtbar geworden.\\
Technologische Fortschritte in Bereichen wie Robotik, Biotechnologie oder künstlicher Intelligenz schaffen schnell neue Möglichkeiten und machen deren Komplexität kaum vorhersehbar. Klimaveränderungen und deren Folgen wie Wassermangel oder extreme Wetterereignisse beeinflussen Logistik und Investitionsentscheidungen. Zudem können neue Zollvorschriften lukrative Geschäftsmodelle von einem Tag auf den anderen zunichtemachen.\\
Diese Dinge führen zu einer Schnelllebigkeit, wodurch es Unternehmen schwerfällt, mit klassischen Methoden und starren Geschäftsmodellen ihre Marktposition zu halten. Daher mussten neue Arbeitsweisen gefunden werden, um sich kontinuierlich und schnell an Veränderungen anpassen zu können.\\
Neben diesen externen Faktoren gibt es auch noch interne Gründe für die Einführung neuer Arbeitsweisen.
Ein wesentlicher interner Treiber ist der Wertewandel im Arbeitsleben. Arbeitnehmer haben den Wunsch eigenverantwortlich Aufgaben zu übernehmen und für den Innovationserfolg des Unternehmens beizutragen.\\
 Agile Methoden wurden daher entwickelt, um sowohl den äußeren Anforderungen als auch den veränderten Erwartungen ihrer Mitarbeitenden gerecht zu werden.%quelle beide bücher

Aktuell werden agile Arbeitsweisen zunehmend hinterfragt und teilweise als ineffektiv und unattraktiv wahrgenommen.\\
Das Ziel dieser Arbeit ist es, die Gründe für diese negative Wahrnehmung zu untersuchen und zu erläutern, warum immer weniger Mitarbeitende bereit sind, agile Methoden anzuwenden. Es wird dabei auf die unterschiedlichen Faktoren eingegangen, welche zur Ablehnung im Umgang mit agilen Methoden führen. Anschließend werden die Auswirkungen des Widerstands gegen agiles Arbeiten erläutert. Abschließend werden Lösungsvorschläge zur Verbesserung der Wahrnehmung und Effizienz von agilen Arbeitsmethoden vorgestellt.



% Jede Section am besten mit einem Kommentar hier im Quelltext markieren
\section{Grundlagen}
\subsection{Was Agilität bedeutet}
Das Thema Agilität und agile Methoden reichen bis in die 1950er Jahre zurück. Damals ist IBM beim Mercury-Projekt der NASA in der Softwareentwicklung inkrementell vorgegangen, die Software wurde also durch kleinste Iterationen kontinuierlich verbessert. Seit 1980 wurden agile Modelle wie Spiral oder Extreme Programming(XP) verwendet.\\
Dies zeigt, dass agile Methoden nicht neu sind und aus verschiedenen Disziplinen geprägt wurde, auch wenn häufig der Eindruck erweckt wird, dass Agilität erst seit dem Agilen Manifest im Jahr 2001 existiert.\\\\%agilität was ist das
Das Agile Manifest wurde von 17 Softwareentwicklern erstellt, um die agile Softwareentwicklung zu leiten. Mittlerweile ist das Agile Manifest auch außerhalb der Softwareentwicklung sehr bekannt.
Das Agile Manifest enthält vier zentrale Werte, wobei jeder Wert durch jeweils zwei Wertepaare abgebildet wird . Dabei wird das erste Wertepaar höher geschätzt als das Zweite. Das bedeutet aber nicht, dass die zweiten Wertepaare unwichtig sind. Im Folgenden werden die vier Grundwerte beschrieben und erläutert.

\textbf{Individuen und Interaktion} mehr als Prozesse und Werkzeuge\\
Der Fokus liegt auf den Menschen und der Zusammenarbeit untereinander. Ein persönliches Gespräch ist dabei wertvoller als ein gut dokumentierter Prozess. Dies lässt sich fördern, indem alle Projektmitglieder im selben Raum sitzen.

\textbf{Funktionierende Software} mehr als Umfasssende Dokumentation\\
Der Schwerpunkt liegt auf der Erstellung von funktionierender Software, weil diese der Mehrwert liefert. Dokumentation soll dabei aber nicht abgeschafft werden, sondern unterstützend eingesetzt werden.

\textbf{Zusammenarbeit mit dem Kunden} mehr als Vertragsverhandlung\\
Eine gute Zusammenarbeit mit dem Kunden ist wichtiger als ein formaler und wasserdichter Vertrag, weil der Kunde durch seine Bedürfnisse ein Teil des Prozesses ist. Ein Vertrag vor Projektbeginn ist nicht in der Lage, alle Situationen im Voraus zu berücksichtigen. Durch eine gemeinsame Abstimmung können Änderungen während der Umsetzung effektiv und gewinnbringend eingeführt werden.

\textbf{Reagieren auf Veränderungen} mehr als das Befolgen eines Plans\\
Ein Plan kann nur so gut sein, wie der Wissenstand zu diesem Zeitpunkt es erlaubt. Da sich Wissen und Rahmenbedingungen kontinuierlich weiterentwickeln, ist es wichtig, auf Veränderungen reagieren zu können. Pläne sollen nicht abgeschafft werden, sollten bei Veränderungen jedoch angepasst werden können.\\


\subsection {Agile Methoden}
\subsubsection{Scrum}
Die am weitesten verbreitete agile Arbeitsmethode ist Scrum. Scrum ist ein Rahmenmodell, damit individuelle Rahmenbedingungen berücksichtigt werden können. Scrum ist einfach und besteht aus wenigen Regeln. Es besteht aus drei Rollen, vier Events und drei Artefakten.\cite{Gaida.2021, Simschek.2022}\\\\
In Scrum gibt es zeitlich begrenzte Ereignisse, die Events. Der Sprint bildet das Herzstück von Scrum. Er ist ein klar begrenzter Zeitraum zwischen zwei bis vier Wochen, in dem ein fertiges, nutzbares Produktinkrement erstellt wird. Im Folgenden werden die vier Events vorgestellt, die ein Sprint umfasst.\\ Im \textbf{Sprint Planning} findet die Planung für den nächsten Sprint statt, also welche Anforderungen umgesetzt werden sollen. Das \textbf{Sprint Review} findet am Ende eines Sprints statt. Dabei wird das Arbeitsergebnis erfasst. Die \textbf{Sprint Retrospektive} findet nach dem Sprint Review und vor dem nächsten Sprint Planning statt. Dabei analysiert das Scrum Team den Ablauf des letzten Sprints, um Verbesserungsmöglichkeiten für den nächsten Sprint zu identifizieren und umzusetzen. Das \textbf{Daily Scrum} ist ein tägliches 15-minütiges Meeting, um die Fortschritte und Hindernisse der Arbeit zu besprechen.\\\\
Scrum besteht aus den Rollen \textbf{Product Owner}, \textbf{Scrum Master} und \textbf{Entwicklungsteam}. Der Product Owner ist für die Priorisierung der Aufgaben verantworlich. Basierend auf den Kundenanforderungen entscheidet er, was im nächsten Sprint umgesetzt werden soll. Der Scrum Master unterstützt das Entwicklungsteam, indem er allen Beteiligten hilft, Scrum zu verstehen und umzusetzen. Der Scrum Master moderiert die Events. Das Entwicklungsteam erledigt die eigentliche Arbeit und ist selbstorganisiert. Außerdem ist das Entwicklungsteam dafür verantwortlich, dass am Ende eines Sprints ein fertiges Inkrement übergeben werden kann.\cite{Mucke.2024}\\\\
Die Artefakte bei Scrum liefern Informationen zu Arbeit und Fortschritt und sind so definiert, dass sie Transparenz bieten, sowie Möglichkeiten zur Überprüfung und Anpassung erlauben.\\ Das \textbf{Produkt Backlog} enthält alle Aufgaben und Anforderungen eines Produkts und wird vom Product Owner verantwortet. Das Product Backlog ist nie final, weil bei der Entwicklung des Produkts neue Anforderungen, Verbesserungen oder Fehler bekannt werden, die bearbeitet werden müssen. Das \textbf{Sprint Backlog} enthält eine Teilmenge der Aufgaben des Product Backlogs. Diese Aufgaben sollen innerhalb dieses Sprints umgesetzt werden.\\ Das \textbf{Inkrement} ist das Ergebnis der im Sprint umgesetzten Arbeit, welches die zuvor erstellten Inkremente erweitert.\\\\
Scrum ist auf kleine Teamgrößen ausgelegt. Das Scrum Team ist klein genug, um flexibel zu bleiben, aber gleichzeitig groß genug, um innerhalb eines Sprints bedeutsame Arbeit zu leisten. In der Regel besteht es aus zehn Personen oder weniger.\cite{Mucke.2024}\\
Ein Merkmal von Scrum ist die klare Abgrenzung der Verantwortlichkeiten. Außenstehende können Einfluss auf das Projekt nehmen, indem sie ihre Änderungswünsche dem Product Owner mitteilen. Es werden innerhalb dieses Sprints aber keine Änderungen an den Anforderungen für diesen Zeitraum vorgenommen, weil das das Entwicklungsteam stören würde. Der Product Owner nimmt die Anforderungen in das Product Backlog auf, welche dann in späteren Sprints bearbeitet werden.\cite{Gluck.2022}


\subsubsection{Kanban}
Kanban ist eine agile Projektmanagementmethode, die heutzutage in der Softwareentwicklung verwendet wird. Ursprünglich wurde Kanban von einem Toyota-Ingenieur entwickelt, um das Produktionssytem von Toyota zu verbessern. Dabei orientierte man sich an der tatsächlichen Nachfrage, anstatt wie zuvor Produkte auf Grundlage geschätzter Bedarfszahlen zu produzieren.\cite{Simschek.2022}

%https://asana.com/de/resources/what-is-kanban
Das zentrale Element von Kanban ist das Kanban Board. Dieses ist eine Projekttafel, die aus Spalten besteht. Die Spalten repräsentieren die verschiedenen Phasen eines Arbeitsprozesses. Einfache Kanban Boards haben Spalten wie \glqq To-Do \grqq\: , \glqq In Arbeit \grqq\: und \glqq Erledigt \grqq\: . Die Anzahl der Spalten kann flexibel an die Anforderungen der Produktion angepasst werden.  In der Softwareentwicklung können beispielsweise zusätzliche Spalten wie \glqq In Testung \grqq\: oder \glqq Blockiert\grqq\: integriert werden. Jede Aufgabe wird als visuelle Karte dargestellt, die zwischen den Spalten verschoben werden kann. Beim Anlegen einer Aufgabe wird die Karte in der Spalte "To Do" platziert und einem oder mehreren Beteiligten zugewiesen, die Karte bewegt sich Schritt-für-Schritt nach rechts bis die Aufgabe erledigt ist.

%https://refa.de/service/refa-lexikon/kanban
In diesem Abschnitt werden sechs Praktiken vorgestellt, welche beachtet werden sollten, um die Vorteile von Kanban nutzen zu können.
\begin{enumerate}
\item Alle Vorgehensweisen müssen transparent kommuniziert werden, damit alle Beteiligten sie verstehen und umsetzen können.
\item Die Anzahl der Karten auf dem Kanban Board muss überschaubar sein.
\item Es müssen immer Karten in der Spalte \glqq In Arbeit\grqq\: vorhanden sein, also immer etwas bearbeitet werden.
\item Alle Kanban Prozesse sollen immer wieder hinterfragt und analysiert werden, um unproduktive Vorgehensweisen aufzudecken und zu beseitigen, um die Effizienz zu erhöhen.
\item Vorgesetzte müssen die Beteiligten so führen, dass sich diese für das Aufrechterhalten des Workflows verantwortlich fühlen und sich für die Optimierung von Abläufen einsetzen.
\item Die grafische Darstellung von Abläufen hilft, Prozesse besser zu verstehen und mögliche Lösungswege aufzuzeigen.
\end{enumerate}

Kanban ermöglicht durch das Kanban Board eine Transparenz, jeder im Team weiß, wer gerade woran arbeitet und wie der Stand der Aufgaben ist. Außerdem hilft das Kanban Board dabei, den Überblick zu behalten und somit auch Engpässe aufzuspüren. Ein weiterer Vorteil ist die flexible Anpassung an die Kundenbedürfnisse, was die Kundenzufriedenheit erhöht. Die einzelnen Bedürfnisse, sei es die Entwicklung eines neuen Produkts oder eine Änderung, werden nach den Anforderungen des Kunden erstellt und als neue Karte auf dem Kanban-Board hinzugefügt.




\section{Gründe für das Scheitern mit agilen Methoden}

Obwohl agile Methoden wie Scrum und Kanban viele Vorteile bieten, scheitern deren Implementierungen in der Praxis häufig aus verschiedenen Gründen. 
Einer der Hauptfaktoren ist der Widerstand gegen Veränderungen, insbesondere in etablierten Organisationsstrukturen, die auf hierarchischen Entscheidungsprozessen basieren. 
Zudem fehlt es oft an ausreichendem Verständnis der agilen Prinzipien und deren konsequenter Anwendung, was zu einer fehlerhaften Umsetzung führt. 
Ein weiterer kritischer Punkt ist die unzureichende Unterstützung durch das Management, wodurch die notwendigen Ressourcen oder die notwendige Kultur der Eigenverantwortung und Kollaboration nicht gefördert werden. 
Auch unrealistische Erwartungen, wie die sofortige Verbesserung von Ergebnissen ohne Berücksichtigung der Lernkurve, tragen zum Scheitern bei. Schließlich können fehlende Kommunikation und unklare Zielsetzungen innerhalb des Teams die Effizienz und den Erfolg agiler Methoden erheblich beeinträchtigen.

\subsection{Fake Agile: Falsche Umsetzung und ihre Folgen}

Agiles Arbeiten verspricht Flexibilität, Anpassungsfähigkeit und Kundenzentrierung. Doch diese Prinzipien können bei falscher Umsetzung verwässert oder sogar ins Gegenteil verkehrt werden – ein Phänomen, das als \textit{Fake Agile} bezeichnet wird. Analog zu \glqq Fake News\grqq{} handelt es sich hierbei um Ansätze, die zwar den Anschein agiler Praktiken erwecken, aber weder deren Kern noch deren Prinzipien verstehen und umsetzen. \textit{Fake Agile} zeigt sich häufig in Form von Cargo-Kult-Methoden, bei denen äußere Merkmale agiler Methoden nachgeahmt werden, ohne die zugrunde liegende Philosophie zu adaptieren.

Ein prägnantes Beispiel ist das Konzept des \textit{Cargo Cult Science}, wie es von dem Physiker Richard Feynman beschrieben wurde. Hier wird der äußere Schein – beispielsweise durch Stand-up-Meetings oder Scrum-Boards – nachgebildet, ohne die grundlegenden Werte wie iteratives Lernen, Transparenz und Teamverantwortung zu verstehen. Diese Oberflächenorientierung führt zu ineffizienten Prozessen, die wenig zur Wertschöpfung beitragen und oft nur Frustration bei den Beteiligten erzeugen.

Häufig scheitern solche Implementierungen an Missverständnissen: Manche Unternehmen erwarten von agilen Methoden eine universelle Lösung aller Probleme. Andere wiederum übertragen die Verantwortung unreflektiert auf Teams, ohne die notwendige Struktur, Kommunikation oder Führungsunterstützung bereitzustellen. Nicht selten entsteht dabei ein organisatorischer Wildwuchs, der mehr Chaos als Innovation fördert. In extremen Fällen können sich isolierte Teamstrukturen entwickeln, die sich gegen die Gesamtorganisation abschotten und gefährliche Dynamiken fördern.

Um \textit{Fake Agile} zu vermeiden, bedarf es einer kritischen Reflexion über die eigenen Zielsetzungen, ein tiefes Verständnis der agilen Prinzipien und eine kontinuierliche Anpassung auf Grundlage empirischer Erkenntnisse. Unternehmen sollten wissenschaftliche Methoden wie Datenanalysen und Simulationen nutzen, um die Effektivität ihrer agilen Maßnahmen regelmäßig zu überprüfen. Nur so kann aus agiler Methodik eine nachhaltige und sinnvolle Praxis werden, die tatsächlich zur Resilienz und Wettbewerbsfähigkeit eines Unternehmens beiträgt.
\cite{Mucke.2024}

\subsection{Kommerzialisierung der Agilität}

Ein wesentlicher Grund für das Scheitern mit agilen Methoden ist die zunehmende Kommerzialisierung der Agilität, auch bekannt als „agil-industrieller Komplex“. Dieser Begriff beschreibt die Geschäftsmacherei rund um agile Methoden, insbesondere Scrum. Organisationen wie Scrum.org oder die Scrum Alliance haben ein profitables Geschäftsmodell entwickelt, das auf Zertifikaten, Workshops und Train-the-Trainer-Programmen basiert. Dabei wird nicht die Förderung agiler Prinzipien in den Fokus gestellt, sondern der Umsatz durch vermeintlich essenzielle Schulungen und Zertifikate.

Zertifizierungen nehmen dabei eine zentrale Rolle ein. Sie vermitteln den Eindruck, dass nur zertifizierte Fachkräfte agile Methoden erfolgreich anwenden können. Dies führt jedoch oft zu einem bürokratischen Formalismus, der die tatsächlichen Werte der Agilität verdrängt. Zwar ändern sich die Inhalte solcher Zertifikate selten, doch deren regelmäßige Erneuerung wird kostenpflichtig verlangt, was eine Scheinwelt schafft, in der formale Nachweise über tatsächliches Verständnis und Kompetenz gestellt werden.

Ein weiteres Problem ist die Vermarktung agiler Methoden als universelles Allheilmittel. Scrum und ähnliche Ansätze werden von Coaches und Beratungsunternehmen oft als Lösungen für sämtliche organisatorischen Herausforderungen angepriesen. Diese Darstellung verkennt jedoch, dass Agilität nur dann erfolgreich sein kann, wenn sie auf die spezifischen Bedürfnisse und Gegebenheiten eines Unternehmens abgestimmt wird. Die daraus resultierenden überzogenen Erwartungen führen bei Misserfolgen häufig zu Frustration und wachsender Skepsis gegenüber agilen Ansätzen.

Die Kommerzialisierung hat dazu geführt, dass sich viele Unternehmen stärker auf die äußeren Symbole und Formalitäten agiler Methoden konzentrieren als auf deren eigentliche Umsetzung. Statt in die Entwicklung einer agilen Unternehmenskultur oder die Verbesserung der Zusammenarbeit zu investieren, wird oft unverhältnismäßig viel Geld für Zertifikate und externe Schulungen ausgegeben. Diese oberflächliche Implementierung lässt die grundlegenden Werte wie Iteration, Selbstorganisation und Kundenorientierung in den Hintergrund treten.

Um der Kommerzialisierungsfalle zu entkommen, sollten Unternehmen kritisch mit den Angeboten des Marktes umgehen und sich auf die tatsächlichen Bedürfnisse ihrer Organisation fokussieren. Es gilt, die Eigenverantwortung der Teams zu stärken und agile Prinzipien praxisnah umzusetzen, anstatt auf formale Nachweise zu vertrauen. Zudem ist es wichtig, sich nicht von Trends oder unrealistischen Versprechungen leiten zu lassen, sondern Agilität als Werkzeug zu verstehen, das kontinuierliche Anpassung und kritische Reflexion erfordert.

Die Kommerzialisierung zeigt, wie leicht eine sinnvolle Methodik durch den Fokus auf Profit und Formalitäten entwertet werden kann. Nur durch eine Rückbesinnung auf die grundlegenden Werte und eine maßvolle, pragmatische Anwendung kann Agilität ihren ursprünglichen Zweck erfüllen und nachhaltig zur Wettbewerbsfähigkeit von Unternehmen beitragen.
\cite{heise.2024}

\section{Auswirkungen}
%- Auf die Teamdynamik (Frustration, Demotivation, Konflikte)
%- Auf Unternehmen (Fehlende Effizienz, hohe Kosten, sinkende Wettbewerbsfähigkeit)
%- Auf die Arbeitskultur (Vertrauensverlust in moderne Methoden, Rückkehr zu traditionellen Ansätzen)

\section{Alternativen und Lösungsansätze}
%- Hybride Ansätze

% und schon der letzte Abschnitt
\section{Fazit}
%- Zusammenfassung der wichtigsten Erkenntnisse
%- Kritische Reflexion: Zukunft agiler Arbeitsweisen und ihr Potenzial


% Bibliographie entweder direkt hier eingeben (nur im Notfall)...
%\begin{thebibliography}{9}
%\bibitem{ACM2019}
%ACM.
%\newblock How to classify works using ACM's computing classification system.
%\newblock \url{http://www.acm.org/class/how_to_use.html}.
%
%\bibitem{Ivory2001}
%M.~Y. Ivory and M.~A. Hearst.
%\newblock The state of the art in automating usability evaluation of user
%  interfaces.
%\newblock {\em ACM Comput. Surv.}, 33(4):470--516, 2001.
%
%\end{thebibliography}

% ... oder die Bibliographie mit Hilfe von BibTeX generieren,
% dies ist auf jeden Fall die bessere Lösung und sollte nach
% Möglichkeit immer verwendet werden:
\bibliographystyle{abbrv}
\bibliography{literatur} % Daten aus der Datei literatur.bib verwenden.

\end{document}
